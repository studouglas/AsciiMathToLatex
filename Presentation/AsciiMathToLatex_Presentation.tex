\documentclass{beamer}
 
\usepackage[utf8]{inputenc}
\usepackage{amsmath}

\setbeamertemplate{navigation symbols}{}
\usetheme{Luebeck}

%-------------------------------------------------------
%	TITLE PAGE
%-------------------------------------------------------
\title[AsciiMath to LaTeX]{AsciiMath to LaTeX}
\subtitle{A Compiler Written in Swift}
\institute{McMaster University}
\author{Stuart Douglas}
 
\begin{document}

\frame{\titlepage}

\begin{frame}
\frametitle{Overview}
\tableofcontents
\end{frame}

% =================== Section =================== 
\section{Introduction} 
\subsection{Brief Project Overview}
\begin{frame}
\frametitle{Brief Project Overview}
\textbf{Motivation}: Math syntax in \LaTeX{} is verbose, difficult-to-read, and hard-to-type\\\vspace{3mm}

\textbf{Solution}: Existing language \emph{AsciiMath} implements subset of \LaTeX{}  math mode in a cleaner, more readable syntax\\\vspace{3mm}

\textbf{Project}: Write a compiler to convert AsciiMath text into \LaTeX{}
\end{frame}

\subsection{Examples}
\begin{frame}
\frametitle{AsciiMath Syntax -- Simple Example}
\end{frame}

\begin{frame}
\frametitle{AsciiMath Syntax -- Complex Example}
\begin{block}{AsciiMath}
\texttt{a/b -= alpha\_(d in RR)\^{}42 \~{}= qz sqrt5}
\end{block}

\begin{block}{\LaTeX{}}
\texttt{\textbackslash frac\{a\}\{b\} \textbackslash{}equiv \textbackslash{}alpha\_\{d \textbackslash{}in \textbackslash{}mathbb\{R\}\}\^{}\{42\} \textbackslash{}cong qz \textbackslash{}sqrt\{5\}}
\end{block}
a/b -= alpha_(d in RR)^42 ~= qz sqrt5
\vspace{4mm}
\textbf{Output}: $a + b < c$

\end{frame}

% =================== Section =================== 
\section{Implementation}
\subsection{Overview}
\subsection{Source Code Example}

% =================== Section =================== 
\section{Demonstration}
\subsection{Demonstration}
\begin{frame}
\frametitle{Demonstration}
\begin{center}
\emph{Demonstration of AsciiMath to \LaTeX}
\end{center}
\end{frame}
\end{document}
